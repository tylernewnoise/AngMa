\documentclass[a4paper,10pt]{article}
\usepackage[ngerman]{babel}		%dt. Übersetzung und Umlaute
\usepackage[utf8]{inputenc}		%Umlaute direkt eingeben
\usepackage{mathtools}			%Mathekrams
\usepackage{paralist}			%bessere Listen
\usepackage{amssymb}			%Mathesymbole
\usepackage{amsthm}				%typesetting theorems (Text über = u.ä.)
\usepackage{fancyhdr} 			%Headerstyles
\usepackage{colortbl}			%Farben in Tabellen
\usepackage{hhline}				%Rahmen in farbigen Cellen
\usepackage[margin=2.0cm,headheight=40pt,top=3cm]{geometry}
\pagestyle{fancy}

\renewcommand{\headrulewidth}{0.4pt}
\renewcommand{\footrulewidth}{0.4pt}
\lhead{Blatt 04 - Teil 1}
\rhead{Angewandte Mathematik}
\cfoot{}
\rfoot{\thepage}
\begin{document}
	\parindent0pt
	\textbf{Aufgabe 01}\\
		$y' = \dfrac{xy^3}{\sqrt{1 + x^2}}$\\
	\begin{align*}
	\dfrac{y'}{y^3} & = \dfrac{x}{\sqrt{1 + x^2}}\\
	\int \dfrac{y'}{y^3}dx & = \int \dfrac{x}{\sqrt{1 + x^2}}dx\\
	\end{align*}
	Substitution von $\int \dfrac{y'}{y^3}dx$:
	\[\int \dfrac{y'}{y^3} = \int \dfrac{1}{y^3}dy \text{ mit } f(w) = \dfrac{1}{w^3} \]
	\begin{align*}
	-\dfrac{1}{2y^2} + c_1 & = \int \dfrac{x}{\sqrt{1 + x^2}}dx
	\end{align*}
	Substitution von $\int \dfrac{x}{\sqrt{1 + x^2}}dx$:
	\begin{align*}
	z(x)  & = 1 + 2x^2\\
	\dfrac{dz}{dx} = 4x & \Leftrightarrow dx = \dfrac{dz}{4x}\\
	\int \dfrac{x}{\sqrt{z}}\dfrac{dz}{4x} & = \dfrac{1}{4} \int \dfrac{1}{\sqrt{z}}dz\\
	\dfrac{1}{4} \int \dfrac{1}{\sqrt{z}}dz & = \dfrac{1}{4} \int z^{-\frac{1}{2}}dz\\
	\dfrac{1}{2} \cdot 2 \sqrt{z} & = \dfrac{\sqrt{z}}{2}
	\end{align*}
	$z(x)$ einsetzen:
	\begin{align*}
	-\dfrac{1}{2y^2} + c_1  & = \dfrac{\sqrt{1 + 2x^2}}{2} + c_2 \\
	-\dfrac{1}{2y^2} & = \dfrac{\sqrt{1 + 2x^2}}{2} + c_3 & \text{ mit } c_3 = c_2 - c_1 \\
	-\dfrac{1}{2} & = \Big (\dfrac{\sqrt{1 + 2x^2}}{2} + c_3 \Big)y^2\\
	y^2 & = \dfrac{-\frac{1}{2}}{\frac{\sqrt{1 + 2x^2}}{2} + c_3}\\
	y^2 & = - \dfrac{1}{2 (\frac{\sqrt{1 + 2x^2}}{2} + c_3)}\\
	y^2 & = - \dfrac{1}{\sqrt{1 + x^2} + 2c_3}
	\end{align*}
	Vollständige Lösung: $y = \sqrt[\pm]{\dfrac{1}{-2c_3 - \sqrt{1 + x^2}}} \quad \text{ mit } 2c_3 + \sqrt{2x^2 + 1} \neq 0$ 
	\newpage
	Somit ergibt sich für das Anfangswertproblem $y(0) = -1$ :
	\begin{align*}
	-1 & = \sqrt[\pm]{\dfrac{1}{-2c_3 - \sqrt{1 + 0^2}}}\\
	-1 & = \sqrt[\pm]{\dfrac{1}{-2c_3 - 1 }}\\
	\end{align*}
	Fall 1:
	\begin{align*}
	1 & = \dfrac{1}{-2c_3 - 1}\\
	1 & = -2c_3 - 1 = 1\\
	2c_3 & = -2\\
	c_3 & = -1
	\end{align*}
	\begin{center}
		$\underline{\underline{y_1 = \sqrt{\dfrac{1}{2 - \sqrt{2x^2 + 1}}}}}$
	\end{center}
	
	Fall 2:
	\begin{align*}
	1 & = \dfrac{1}{2c_3 + 1}\\
	1 & = 2c_3 +1\\
	c_3 & = 0
	\end{align*}
	\begin{center}
		$\underline{\underline{y_2 = \sqrt{\dfrac{1}{-\sqrt{2x^2 + 1}}}}}$
	\end{center}
	\newpage
	\textbf{Aufgabe 02}\\
	$y' + (2x - 1)y = xe^x$\\\\
	Homogenen Teil l"osen:\\
	\begin{align*}
		& y' + (2x - 1) \cdot y   =  0\\
		\Leftrightarrow \quad & \dfrac{y'}{y}  =  -2x + 1	\\
		\Leftrightarrow \quad & \int \dfrac{y'}{y}dx  =  \int-2x + 1 dx \\
		\Leftrightarrow \quad & \int \dfrac{1}{y}dy  =  -x^2 + x +c_2\\
		\Leftrightarrow \quad & \text{ln}(y) + c_1  =  -x^2 + x +c_2 \\
		\Leftrightarrow \quad & \text{ln}(y)  =  -x^2+ x + c_3  & \text{ mit }c_3 = c_2 - c_1 \\
		\Leftrightarrow \quad & e^{\text{ln}(y)}  = e^{(-x^2+x)}c_4  & \text{ mit } c_4 = e^{c_3} \\
		\Leftrightarrow \quad & y = e^{-x^2 + x}c_4
	\end{align*}
	Die vollst"andige L"osung lautet: $y = e^{-x^2 + x}c_4$.\\
		
	Ersetze nun $c$ mit $c(x)$: $\quad y = e^{-x^2 + x} \cdot c(x)$\\
	
	Ableiten:
	\begin{align*}
		y' & = e^{(-x^2 + x)} \cdot (-2x + 1) \cdot c(x) + e^{(-x^2 + x)} \cdot c'(x)\\
		y' & = (-2x +1) \cdot y + e^{(-x^2 + x)} \cdot c'(x)\\
		y' & = -(2x +1) \cdot y + e^{(-x^2 + x)} \cdot c'(x)\\
		y' + (2x +1) \cdot y & = e^{(-x^2 + x)} \cdot c'(x)
	\end{align*}
	Inhomogenen Teil einsetzen: \[xe^x = e^{(-x^2 + x)} \cdot c'(x)\]
	
	$c(x)$ berechnen:
	\begin{align*}
		c'(x) & = \dfrac{xe^x}{e^{(-x^2 + x)}} = \dfrac{xe^x}{e^{-(x^2 + x)}} \\
		c'(x) & = x\cdot e^x \cdot e^{(x^2 - x)} \\
		c'(x) & = x\cdot e^{x^2}\\
		\int c'(x) dx & = \int x \cdot e^{x^2} dx\\
		c(x) & = \dfrac{e^{x^2}}{2} + c
	\end{align*}

	In $y$ einsetzen:
	\begin{align*}
	y & = e^{(-x^2 + x)} \cdot \Big( \dfrac{e^{x^2}}{2} + c\Big)
	\end{align*}
	\begin{center}
		$\underline{\underline{y = \dfrac{1}{2} \cdot e^x + c \cdot e^{(-x^2 + x)}}}$
	\end{center}



\end{document}
