\documentclass[a4paper,10pt]{article}
\usepackage[ngerman]{babel}		%dt. Übersetzung und Umlaute
\usepackage[utf8]{inputenc}		%Umlaute direkt eingeben
\usepackage{mathtools}			%Mathekrams
\usepackage{paralist}			%bessere Listen
\usepackage{amssymb}			%Mathesymbole
\usepackage{amsthm}				%typesetting theorems (Text über = u.ä.)
\usepackage{fancyhdr} 			%Headerstyles
\usepackage{colortbl}			%Farben in Tabellen
\usepackage{hhline}				%Rahmen in farbigen Cellen
\usepackage{multicol}
\usepackage[margin=2.0cm,headheight=40pt,top=3cm]{geometry}
\pagestyle{fancy}

\renewcommand{\headrulewidth}{0.4pt}
\renewcommand{\footrulewidth}{0.4pt}
\lhead{Blatt 01 - Teil 2}
\rhead{Angewandte Mathematik}
\cfoot{}
\rfoot{\thepage}
\begin{document}
	\parindent0pt
	\textbf{Aufgabe 03}\\
	Zusammenfassung:
	\begin{itemize}
		\item 
		Produkt 1: $P_1$\\
		Produkt 2: $P_2$
		\item Maschinen $M_1$ und $M_2$ laufen 12h und 16h und ben"otigen 2h und 2h bzw. 4h zur Herstellung eines Produkts:\\
		\begin{align*}
		\text{Sei: } & x_1 := P_1 \text{ an } M_1\\
		& x_2 := P_1 \text{ an } M_2\\
		& y_1 := P_2 \text{ an } M_1
		\end{align*}
		\item Gewinn soll maximiert werden: $\mathrm{max}(2000(x_1 + x_2) + 3000y_1)$
		\item 20 Einheiten an Material sind vorhanden, wobei $P_1$ 2 Einheiten und $P_2$ 4 Einheiten an den Maschinen ben"otigt:
		\[
		\begin{matrix}
		2(x_1 + x_2) + 4y_1 \leq 20\\
		2x_1 + 2y_1 \leq 12\\
		4x_2 \leq 16
		\end{matrix}
		\]
	\end{itemize}
	Daraus ergibt sich:
	\[
	(P) \mathrm{max}(2000(x_1 + x_2) + 3000y_1) \mid \begin{matrix}
	2x_1 + 2x_2 + 4y_1 \leq 20\\
	2x_1 + 0x_2 + 2y_1 \leq 12\\
	0x_1 + 4x_2 + 0y_1 \leq 16\\
	x_1, x_2, y_1 \geq 0
	\end{matrix}
	\
	\}
	\]
	Nach Hinzuf"ugen von Schlupfvariablen erhalten wir die folgende Matrix:
	\[
	A:=
	\begin{pmatrix}
	2 & 2 & 4 & 1 & 0 & 0\\
	2 & 0 & 2 & 0 & 1 & 0\\
	0 & 4 & 0 & 0 & 0 & 1\\
	\end{pmatrix} \quad b:= 
	\begin{pmatrix}
	20\\
	12\\
	16
	\end{pmatrix}
	\]
	Daraus ergeben sich die folgenden Simplextableaus:\\
	\[
	\setlength{\extrarowheight}{2pt}
	\begin{tabular}{ccccc}
		\hhline{~~---}
		& \multicolumn{1}{c|}{} & \cellcolor[gray]{0.9}$x_1$ & $x_2$ & $y_1$ \\ \hhline{~----}
		\multicolumn{1}{c|}{} & \multicolumn{1}{c|}{0} & \cellcolor[gray]{0.9}-2000 & -2000 & -3000 \\ \hhline{-----}
		\multicolumn{1}{|c|}{$u_1$} & \multicolumn{1}{c|}{20} & \cellcolor[gray]{0.9}2 & 2 & 4\\ 
		\multicolumn{1}{|c|}{\cellcolor[gray]{0.9}$u_2$} & \multicolumn{1}{c|}{\cellcolor[gray]{0.9}12} & 2 & \cellcolor[gray]{0.9}0 & \cellcolor[gray]{0.9}2\\ 
		\multicolumn{1}{|c|}{$u_3$} & \multicolumn{1}{c|}{16} & \cellcolor[gray]{0.9}4 & 4 & 0\\
		\multicolumn{5}{c}{$B^+ = \{u_1, u_2, u_3\}$} 
	\end{tabular}
	\setlength{\extrarowheight}{0pt}
	\quad
	x^* = 
	\begin{pmatrix}
	x_1\\
	x_2\\
	y_1
	\end{pmatrix}
	=
	\begin{pmatrix}
	0\\
	0\\
	0
	\end{pmatrix}
	\quad\quad
	\setlength{\extrarowheight}{2pt}
	\begin{tabular}{ccccc}
	\hhline{~~---}
	& \multicolumn{1}{c|}{} & $u_2$ & \cellcolor[gray]{0.9}$x_2$ & $y_1$ \\ \hhline{~----}
	\multicolumn{1}{c|}{} & \multicolumn{1}{c|}{12000} & 1000 & \cellcolor[gray]{0.9}-2000 & -1000\\ \hhline{-----}
	\multicolumn{1}{|c|}{\cellcolor[gray]{0.9}$u_1$} & \multicolumn{1}{c|}{\cellcolor[gray]{0.9}8} & \cellcolor[gray]{0.9} -1 & 2 & \cellcolor[gray]{0.9}2 \\ 
	\multicolumn{1}{|c|}{$x_1$} & \multicolumn{1}{c|}{6} & $\frac{1}{2}$ & \cellcolor[gray]{0.9}0 & 1\\ 
	\multicolumn{1}{|c|}{$u_3$} & \multicolumn{1}{c|}{-8} & -2 & \cellcolor[gray]{0.9}4 & -4\\
	\multicolumn{5}{c}{$B^+ = \{u_1\}$} 
	\end{tabular}
	\setlength{\extrarowheight}{0pt}
	\quad
	x^* = 
	\begin{pmatrix}
	x_1\\
	x_2\\
	y_1
	\end{pmatrix}
	=
	\begin{pmatrix}
	6\\
	0\\
	0
	\end{pmatrix}
	\]
	\\\\
	$
	\setlength{\extrarowheight}{2pt}
	\begin{tabular}{ccccc}
	\hhline{~~---}
	& \multicolumn{1}{c|}{} & $u_2$ & $u_1$ & $y_1$ \\ \hhline{~----}
	\multicolumn{1}{c|}{} & \multicolumn{1}{c|}{20000} & 0 & 1000 & 1000\\ \hhline{-----}
	\multicolumn{1}{|c|}{$x_2$} & \multicolumn{1}{c|}{4} & $-\frac{1}{2}$ & $\frac{1}{2}$ & 1\\ 
	\multicolumn{1}{|c|}{$x_1$} & \multicolumn{1}{c|}{6} & $\frac{1}{2}$ & 0 & 1\\ 
	\multicolumn{1}{|c|}{$u_3$} & \multicolumn{1}{c|}{-24} & 0 & -2 & -8\\
	\end{tabular}
	\setlength{\extrarowheight}{0pt}
	\quad
	x^* = 
	\begin{pmatrix}
	x_1\\
	x_2\\
	y_1
	\end{pmatrix}
	=
	\begin{pmatrix}
	4\\
	6\\
	0
	\end{pmatrix}	
	$\\
	
	Das Tableau ist damit optimal, es kann maximal ein Gewinn von 20000 Euro erzielt werden. Es muss nur das Produkt $P_1$ an den beiden Standorten in den Mengen 4 und 6 hergestellt werden. Produkt $P_2$ ist für eine Gewinnmaximierung nicht notwendig.
	\newpage
	\textbf{Aufgabe 04}
	\begin{compactenum} [(a)]
		\item 
		Nach Hinzuf"ugen von Schlupfvariablen erhalten wir die folgende Matrix:\\
		\[
		A:=
		\begin{pmatrix}
		1 & 1 & 1 & 0 & 0\\
		2 & 1 & 0 & 1 & 0\\
		2 & 5 & 0 & 0 & 1
		\end{pmatrix} \quad b:= 
		\begin{pmatrix}
		10\\
		18\\
		35
		\end{pmatrix}
		\]
		\ \\
		\begin{multicols}{2}
			Nach Auswahl von $x_1$: \\\\
			$
			\setlength{\extrarowheight}{2pt}
			\begin{tabular}{cccc}
			\hhline{~~--}
			& \multicolumn{1}{c|}{}    & \multicolumn{1}{c}{\cellcolor[gray]{0.9}$x_1$} & \multicolumn{1}{c}{$x_2$} \\ \hhline{~---}
			\multicolumn{1}{c|}{}    & \multicolumn{1}{c|}{-3} & \multicolumn{1}{c}{\cellcolor[gray]{0.9}-7}   & \multicolumn{1}{c}{-7} \\ \hline
			\multicolumn{1}{|c|}{$u_1$} & \multicolumn{1}{c|}{10} & \cellcolor[gray]{0.9}1 & 1\\
			\multicolumn{1}{|c|}{\cellcolor[gray]{0.9}$u_2$} & \multicolumn{1}{c|}{\cellcolor[gray]{0.9}18} & 2 & \cellcolor[gray]{0.9}1\\
			\multicolumn{1}{|c|}{$u_3$} & \multicolumn{1}{c|}{35} & \cellcolor[gray]{0.9}2 & 5\\
			\multicolumn{4}{c}{$B^+ = \{u_1, u_2, u_3\}$} 
			\end{tabular}
			\setlength{\extrarowheight}{0pt}
			\quad
			x^* = 
			\begin{pmatrix}
			x_1\\
			x_2
			\end{pmatrix}
			=
			\begin{pmatrix}
			0\\
			0
			\end{pmatrix}
			$
			\\\\\\
			$
			\setlength{\extrarowheight}{2pt}
			\begin{tabular}{ccrr}
			\hhline{~~--}
			& \multicolumn{1}{c|}{}    & \multicolumn{1}{r}{$u_2$} & \multicolumn{1}{r}{\cellcolor[gray]{0.9}$x_2$} \\ \hhline{~---} 
			\multicolumn{1}{c|}{}    & \multicolumn{1}{c|}{60} & \multicolumn{1}{r}{$\frac{7}{2}$} & \multicolumn{1}{r}{\cellcolor[gray]{0.9}$-\frac{7}{2}$} \\ \hline
			\multicolumn{1}{|c|}{\cellcolor[gray]{0.9}$u_1$} & \multicolumn{1}{c|}{\cellcolor[gray]{0.9}1} & \cellcolor[gray]{0.9}$-\frac{1}{2}$ & $\frac{1}{2}$ \\
			\multicolumn{1}{|c|}{$x_1$} & \multicolumn{1}{c|}{9} & $\frac{1}{2}$ & \cellcolor[gray]{0.9}$\frac{1}{2}$ \\
			\multicolumn{1}{|c|}{$u_3$} & \multicolumn{1}{c|}{17} & -1 & \cellcolor[gray]{0.9}4\\
			\multicolumn{4}{c}{$B^+ = \{u_1, x_1, u_3\}$}
			\end{tabular}
			\setlength{\extrarowheight}{0pt}
			\quad
			x^* = 
			\begin{pmatrix}
			x_1\\
			x_2
			\end{pmatrix}
			=
			\begin{pmatrix}
			9\\
			0
			\end{pmatrix}
			$
			\\\\\\
			$
			\setlength{\extrarowheight}{2pt}
			\begin{tabular}{cccc}
			\cline{3-4}
			& \multicolumn{1}{c|}{}    & \multicolumn{1}{c}{$u_2$} & \multicolumn{1}{c}{$u_1$} \\ \cline{2-4}
			\multicolumn{1}{c|}{}    & \multicolumn{1}{c|}{67} & \multicolumn{1}{c}{0}   & \multicolumn{1}{c}{7} \\ \hline
			\multicolumn{1}{|c|}{$x_2$} & \multicolumn{1}{c|}{2} & -1 & 2\\
			\multicolumn{1}{|c|}{$x_1$} & \multicolumn{1}{c|}{8} & 1 & -1\\
			\multicolumn{1}{|c|}{$u_3$} & \multicolumn{1}{c|}{9} & 3 & -8
			\end{tabular}
			\setlength{\extrarowheight}{0pt}
			\quad
			x^* = 
			\begin{pmatrix}
			x_1\\
			x_2
			\end{pmatrix}
			=
			\begin{pmatrix}
			8\\
			2
			\end{pmatrix}
			$
			\columnbreak
			
			Nach Auswahl von $x_2$: \\\\
			$
			\setlength{\extrarowheight}{2pt}
			\begin{tabular}{cccc}
			\hhline{~~--}
			& \multicolumn{1}{c|}{} & \multicolumn{1}{c}{$x_1$} & \multicolumn{1}{c}{\cellcolor[gray]{0.9}$x_2$} \\ \hhline{~---}
			\multicolumn{1}{c|}{} & \multicolumn{1}{c|}{-3} & \multicolumn{1}{c}{-7} & \multicolumn{1}{c}{\cellcolor[gray]{0.9}-7} \\ \hline
			\multicolumn{1}{|c|}{$u_1$} & \multicolumn{1}{c|}{10} & 1 & \cellcolor[gray]{0.9}1\\
			\multicolumn{1}{|c|}{$u_2$} & \multicolumn{1}{c|}{18} & 2 & \cellcolor[gray]{0.9}1\\
			\multicolumn{1}{|c|}{\cellcolor[gray]{0.9}$u_3$} & \multicolumn{1}{c|}{\cellcolor[gray]{0.9}35} & \cellcolor[gray]{0.9}2 & 5\\
			\multicolumn{4}{c}{$B^+ = \{u_1, u_2, u_3\}$} 
			\end{tabular}
			\setlength{\extrarowheight}{0pt}
			\quad
			x^* = 
			\begin{pmatrix}
			x_1\\
			x_2
			\end{pmatrix}
			=
			\begin{pmatrix}
			0\\
			0
			\end{pmatrix}
			$
			\\\\\\
			$
			\setlength{\extrarowheight}{2pt}
			\begin{tabular}{ccrr}
			\hhline{~~--}
			& \multicolumn{1}{c|}{} & \multicolumn{1}{r}{\cellcolor[gray]{0.9}$x_1$} & \multicolumn{1}{r}{$u_3$} \\ \hhline{~---} 
			\multicolumn{1}{c|}{} & \multicolumn{1}{c|}{46} & \multicolumn{1}{r}{\cellcolor[gray]{0.9}$-\frac{21}{5}$} & \multicolumn{1}{r}{$-\frac{7}{5}$} \\ \hline
			\multicolumn{1}{|c|}{\cellcolor[gray]{0.9}$u_1$} & \multicolumn{1}{c|}{\cellcolor[gray]{0.9}3} & $\frac{3}{5}$ & \cellcolor[gray]{0.9}$-\frac{1}{5}$ \\
			\multicolumn{1}{|c|}{$u_2$} & \multicolumn{1}{c|}{11} & \cellcolor[gray]{0.9}$\frac{8}{5}$ & $-\frac{1}{5}$ \\
			\multicolumn{1}{|c|}{$x_2$} & \multicolumn{1}{c|}{7} & \cellcolor[gray]{0.9}$\frac{2}{5}$ & $\frac{1}{5}$ \\
			\multicolumn{4}{c}{$B^+ = \{u_1, u_2, x_2\}$}
			\end{tabular}
			\setlength{\extrarowheight}{0pt}
			\quad
			x^* = 
			\begin{pmatrix}
			x_1\\
			x_2
			\end{pmatrix}
			=
			\begin{pmatrix}
			0\\
			7
			\end{pmatrix}
			$
			\\\\\\
			$
			\setlength{\extrarowheight}{2pt}
			\begin{tabular}{cccc}
			\cline{3-4}
			& \multicolumn{1}{c|}{} & \multicolumn{1}{c}{$u_1$} & \multicolumn{1}{c}{$u_3$} \\ \cline{2-4}
			\multicolumn{1}{c|}{} & \multicolumn{1}{c|}{67} & \multicolumn{1}{c}{7} & \multicolumn{1}{c}{0} \\ \hline
			\multicolumn{1}{|c|}{$x_1$} & \multicolumn{1}{c|}{5} & $\frac{5}{3}$ & $-\frac{1}{3}$\\
			\multicolumn{1}{|c|}{$u_2$} & \multicolumn{1}{c|}{3} & $\frac{8}{3}$ & $\frac{1}{3}$\\
			\multicolumn{1}{|c|}{$x_2$} & \multicolumn{1}{c|}{5} & $\frac{2}{3}$ & $\frac{1}{3}$
			\end{tabular}
			\setlength{\extrarowheight}{0pt}
			\quad
			x^* = 
			\begin{pmatrix}
			x_1\\
			x_2
			\end{pmatrix}
			=
			\begin{pmatrix}
			5\\
			5
			\end{pmatrix}
			$
		\end{multicols}
		Der Wert der Zielfunktion ist 67, die vollständige Lösung ist:
		\[
		\Big \{
		\lambda 
		\begin{pmatrix}
		8\\
		2
		\end{pmatrix}
		+
		(1 - \lambda)
		\begin{pmatrix}
		5\\
		5
		\end{pmatrix}
		\mid
		\lambda \in [0,1] \Big\}
		\]
		\\\\
		\item
		Nach Hinzuf"ugen von Schlupfvariablen erhalten wir die folgende Matrix:\\
		\[
		A:=
		\begin{pmatrix}
		-4 & 1 & 1 & 0 & 0\\
		1 & -1 & 0 & 1 & 0\\
		2 & -1 & 0 & 0 & 1
		\end{pmatrix} \quad b:= 
		\begin{pmatrix}
		4\\
		4\\
		10
		\end{pmatrix}
		\]
		\ \\
		\[
		\setlength{\extrarowheight}{2pt}
		\begin{tabular}{cccc}
		\hhline{~~--}
		& \multicolumn{1}{c|}{}    & \multicolumn{1}{c}{$x_1$} & \multicolumn{1}{c}{\cellcolor[gray]{0.9}$x_2$} \\ \hhline{~---}
		\multicolumn{1}{c|}{}    & \multicolumn{1}{c|}{0} & \multicolumn{1}{c}{-5}   & \multicolumn{1}{c}{\cellcolor[gray]{0.9}-4} \\ \hline
		\multicolumn{1}{|c|}{\cellcolor[gray]{0.9}$u_1$} & \multicolumn{1}{c|}{\cellcolor[gray]{0.9}4} & \cellcolor[gray]{0.9}-4 & 1\\
		\multicolumn{1}{|c|}{$u_2$} & \multicolumn{1}{c|}{4} & 1 & \cellcolor[gray]{0.9}-1\\
		\multicolumn{1}{|c|}{$u_3$} & \multicolumn{1}{c|}{10} & 2 & \cellcolor[gray]{0.9}-1\\
		\multicolumn{4}{c}{$B^+ = \{u_1\}$}
		\end{tabular}
		\setlength{\extrarowheight}{0pt}
		\quad
		x^* = 
		\begin{pmatrix}
		x_1\\
		x_2
		\end{pmatrix}
		=
		\begin{pmatrix}
		0\\
		0
		\end{pmatrix}
		\quad
		\quad
		\setlength{\extrarowheight}{2pt}
		\begin{tabular}{cccc}
		\hhline{~~--}
		& \multicolumn{1}{c|}{}    & \multicolumn{1}{l}{\cellcolor[gray]{0.9}$x_1$} & \multicolumn{1}{c}{$u_1$} \\ \hhline{~---} 
		\multicolumn{1}{c|}{}    & \multicolumn{1}{c|}{16} & \multicolumn{1}{c}{\cellcolor[gray]{0.9}-21} & \multicolumn{1}{c}{4} \\ \hline 
		\multicolumn{1}{|c|}{$x_2$} & \multicolumn{1}{c|}{4} & \cellcolor[gray]{0.9}-4 & 1 \\
		\multicolumn{1}{|c|}{$u_2$} & \multicolumn{1}{c|}{8} & \cellcolor[gray]{0.9}-3 & 1 \\
		\multicolumn{1}{|c|}{$u_3$} & \multicolumn{1}{c|}{14} & \cellcolor[gray]{0.9}-2 & 1\\
		\multicolumn{4}{c}{$B^+ = \{\varnothing\}$}
		\end{tabular}
		\setlength{\extrarowheight}{0pt}
		\]
		Da $B^+ = \{\varnothing\}$ ist kein Pivotelement w"ahlbar und das Problem somit nicht l"osbar.
	\end{compactenum} 
\end{document}
