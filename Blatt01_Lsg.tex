\documentclass[a4paper,10pt]{article}
\usepackage[ngerman]{babel}		%dt. Übersetzung und Umlaute
\usepackage[utf8]{inputenc}		%Umlaute direkt eingeben
\usepackage{mathtools}			%Mathekrams
\usepackage{paralist}			%bessere Listen
\usepackage{amssymb}			%Mathesymbole
\usepackage{amsthm}			%typesetting theorems (Text über = u.ä.)
\usepackage{fancyhdr} 			%Headerstyles
\usepackage{verbatim}			%Sourcecode einfügen
\usepackage[margin=2.0cm,headheight=40pt,top=3cm]{geometry}
\usepackage{tikz}
\usepackage{cancel}
\usepackage{stmaryrd}
\usepackage{colortbl}
\usetikzlibrary{matrix,positioning,arrows, automata}

\pagestyle{fancy}
\renewcommand{\headrulewidth}{0.4pt}
\renewcommand{\footrulewidth}{0.4pt}
\lhead{Blatt 01}
\rhead{Angewandte Mathematik}
\cfoot{}
\rfoot{\thepage}
\begin{document}
	\parindent0pt
	\textbf{Aufgabe 01}\\
	
	\begin{compactenum} [(a)]
		\item \begin{itemize}
			\item $ <v,w> := v_1 * w_1 = w_1 * v_1 =: <w,v> $
			\item $ <v+y,w> := (v+y)_1 * w_1 = (v_1 + y_1) * w_1 = v_1 * w_1 + y_1* w_1 =: <v,w> + <y,w>$
			\item $ < \lambda v,w> := \lambda v_1 * w_1 = \lambda *(v_1 * w_1) =: \lambda * <v,w> $
			\item $ <v,v> := v_1 * v_1 = v_1^2 \geq 0 \qquad \forall v \in \mathbb{R}^n $ (wegen dem Quadrat)
			\item $ <0,0> := 0*0 = 0 $
		\end{itemize}
		$ \Longrightarrow $ Die Abbildung in (a) definiert ein Skalarprodukt\\
	
		\item $<v+y,w> \neq <v,w> + <y,w> $\\\\
		Gegenbeispiel: Wir betrachten die Vektoren\\
		$ \vec{v} = \begin{pmatrix}
		1 \\ 2
		\end{pmatrix} \qquad
		\vec{y} = \begin{pmatrix}
		3 \\ 4
		\end{pmatrix} \qquad
		\vec{w} = \begin{pmatrix}
		5 \\ 1
		\end{pmatrix}$\\
		\\
		$ <v+y,w> := \sqrt{\sum_{i = 1}^{n} (v+y)_i} * \sqrt{\sum_{i = 1}^{n} w_i} = \sqrt{\sum_{i = 1}^{n} \begin{pmatrix}
			4 \\ 6 \end{pmatrix}} * \sqrt{\sum_{i = 1}^{n} \begin{pmatrix}
			5 \\ 1 \end{pmatrix}} = \sqrt{10} * \sqrt{6} \approx 7,746$ \\
		$ <v,w> + <y,w> := \sqrt{\sum_{i = 1}^{n} v_i} * \sqrt{\sum_{i = 1}^{n} w_i} + \sqrt{\sum_{i = 1}^{n} y_i} * \sqrt{\sum_{i = 1}^{n} w_i} = (\sqrt{3} * \sqrt{6}) + (\sqrt{7} * \sqrt{6}) \approx 10,723 $ \\ \\
		$ \Longrightarrow $ Die Abbildung in (b) definiert kein Skalarprodukt \\
		
		\item $ <v+y,w> \neq <v,w> + <y,w> $\\\\
		Gegenbeispiel: Wir betrachten die Vektoren\\
		$ \vec{v} = \begin{pmatrix}
		1 \\ 30
		\end{pmatrix} \qquad
		\vec{y} = \begin{pmatrix}
		1 \\ -1
		\end{pmatrix} \qquad
		\vec{w} = \begin{pmatrix}
		2 \\ 1
		\end{pmatrix}$\\
		$ <v+y,w> = \underset{1 \leq i \leq n}{\text{max}} (
			(v_i+y_i) * w_i) = \underset{1 \leq i \leq n}{\text{max}} (\begin{pmatrix} 2 \\ 29 \end{pmatrix} * 
			\begin{pmatrix} 2 \\ 1 \end{pmatrix}) = 29 $ \\
		$ <v,w> + <y,w> := \underset{1 \leq i \leq n}{\text{max}} (v_i + w_i) + \underset{1 \leq i \leq n}{\text{max}} (y_i * w_i) = 
		\underset{1 \leq i \leq n}{\text{max}} (
		\begin{pmatrix}
		1 \\ 30
		\end{pmatrix} + 
		\begin{pmatrix}
		2 \\ 1
		\end{pmatrix}) + \underset{1 \leq i \leq n}{\text{max}} (
		\begin{pmatrix}
		1 \\ -1
		\end{pmatrix} + 
		\begin{pmatrix}
		2 \\ 1
		\end{pmatrix} )\\
		= 30 + 2 = 32 \\
		\Longrightarrow $ Die Abbildung in (c) definiert kein Skalarprodukt
		
	\end{compactenum} \
	
	\textbf{Aufgabe 02}\\
	\begin{compactenum} [(a)]
		\item TODO
		\item TODO
	\end{compactenum}
	
	
\end{document}
