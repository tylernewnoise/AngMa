\documentclass[a4paper,10pt]{article}
\usepackage[ngerman]{babel}		%dt. Übersetzung und Umlaute
\usepackage[utf8]{inputenc}		%Umlaute direkt eingeben
\usepackage{mathtools}			%Mathekrams
\usepackage{paralist}			%bessere Listen
\usepackage{amssymb}			%Mathesymbole
\usepackage{amsthm}				%typesetting theorems
\usepackage{listings}
\lstset{language=Java}

\usepackage{fancyhdr} 			%Headerstyles
\usepackage[margin=2.0cm,headheight=40pt,top=3cm]{geometry}
\pagestyle{fancy}

\renewcommand{\headrulewidth}{0.4pt}
\renewcommand{\footrulewidth}{0.4pt}
\lhead{Blatt 06 - Teil 1}
\rhead{Angewandte Mathematik}
\cfoot{}
\rfoot{\thepage}
\begin{document}
	\parindent0pt
	\textbf{Aufgabe 01}\\
	\begin{enumerate}[a)]
		\item 
		\item 
		\item 
		\newpage
		\item Lagrange Interpolationspolynom f"ur $\omega(x) = \underset{j \neq k \atop j = 0}{\overset{4}{\prod}}  \dfrac{x - x_j}{x_k - x_j}$\\
		\begin{align*}
			\omega_0(x) & = \dfrac{(x - x_1)(x - x_2)(x - x_3)(x - x_4)}{(x_0 - x_1)(x_0 - x_2)(x_0 - x_3)(x_0 - x_4)} = \dfrac{(x + 1)(x - 0)(x - 1)(x - 2)}{(-2 + 1)(-2 -0)(-2 -1)(-2 -2)}\\
			& = \dfrac{(x^2 + x)(x^2 - 2x - x + 2)}{(-1)(-2)(-3)(-4)} = \dfrac{x^4 - 2x^3 -x^3 + 2x^2 +x^3 -2x^2 -x^2 + 2x}{24}\\
			& = \frac{1}{24}(x^4 - 2x^3 -x^2 + 2x)\\
			\omega_1(x) & = \dfrac{(x - x_0)(x - x_2)(x - x_3)(x - x_4)}{(x_1 - x_0)(x_1 - x_2)(x_1 - x_3)(x_1 - x_4)} = \dfrac{(x + 2)(x - 0)(x - 1)(x - 2)}{(-1 + 2)(-1)(-1 - 1)(-1 - 2)}\\
			& = \dfrac{(x^2 + 2x)(x^2 - 3x + 2)}{(1)(-1)(-2)(-3)} = \dfrac{x^4 - 3x^3 + 2x^2 + 2x^3 - 6x^2 + 4x}{-6}\\
			& = -\dfrac{1}{6}(x^4 - x^3 -4x^2 + 4x)\\
			\omega_2(x) & = \dfrac{(x - x_0)(x - x_1)(x - x_3)(x - x_4)}{(x_2 - x_0)(x_2 - x_1)(x_2 - x_3)(x_2 - x_4)} = \dfrac{(x + 2)(x + 1)(x - 1)(x - 2)}{(0 + 2)(0 + 1)(0 - 1)(0 - 2)}\\
			& = \dfrac{(x^2 + 3x + 2)(x^2 - 3x + 2)}{(2)(1)(-1)(-2)} = \dfrac{(x^4 - 3x^3 + 2x^2 + 3x^3 - 9x^2 + 6x + 2x^2 - 6x + 4)}{4}\\
			& = \dfrac{1}{4}(x^4 - 4x^2 + 4)\\
			\omega_3(x) & = \dfrac{(x - x_0)(x - x_1)(x - x_2)(x - x_4)}{(x_3 - x_0)(x_3 - x_1)(x_3 - x_2)(x_3 - x_4)} = \dfrac{(x + 2)(x + 1)(x)(x - 2)}{(1 + 2)(1 + 1)(1)(1 - 2)}\\
			& = \dfrac{(x^2 + 3x + 2)(x^2 - 2x)}{(3)(2)(1)(-1)} = \dfrac{x^4 - 2x^3 + 3x^3 -6x^2 + 2x^2 - 4x}{-6}\\
			& = -\dfrac{1}{6}(x^4 + x^3 - 4x^2 -4x)\\
			\omega_4(x) & = \dfrac{(x - x_0)(x - x_1)(x - x_2)(x - x_3)}{(x_4 - x_0)(x_4 - x_1)(x_4 - x_2)(x_4 - x_3)} = \dfrac{(x + 2)(x + 1)(x)(x - 1)}{(2 + 2)(2 + 1)(2)(2 - 1)}\\
			& = \dfrac{(x^2 + 3x  + 2)(x^2 - x)}{(4)(3)(2)(1)} = \dfrac{x^4 - x^3 + 3x^3 -3x^2 + 2x^2 - 2x}{24}\\
			& = \dfrac{1}{24}(x^4 + 2x^3 - x^2 - 2x)
		\end{align*}
		$L_4(x) = \overset{4}{\underset{k = 0}{\sum}}f_k \omega(x)$:\\
		\begin{align*}
		 & \dfrac{2}{24}(x^4 - 2x^3 -x^2 + 2x)\\
		 & + -\dfrac{0}{6}(x^4 - x^3 - 4x^2 + 4x)\\
		 & + \dfrac{0}{4}(x^4 - 4x^2 + 4)\\
		 & + -\dfrac{2}{6}(x^4 + x^3 - 4x^2 - 4x)\\
		 & + \dfrac{6}{24}(x^4 + 2x^3 - x^2 - 2x)
		 \end{align*}
		 \[
		 = \dfrac{2}{24}x^4 - \dfrac{4}{24}x^3 - \dfrac{2}{24}x^2 + \dfrac{4}{24}x - \dfrac{2}{6}x^4 - \dfrac{2}{6}x^3 + \dfrac{8}{6}x^2 + \dfrac{8}{6}x + \dfrac{6}{24}x^4 + \dfrac{12}{24}x^3 - \dfrac{6}{24}x^2 - \dfrac{12}{24}x\\
		 \]
		 \[
		 = \underline{\underline{x^2 + x}}
		 \]
	\end{enumerate}
	\newpage
	\textbf{Aufgabe 2}\\
	\begin{enumerate}[a)]
		\item 
		\begin{itemize}
			\item Trapezregel: $\int_{a}^{b} f(x) dx = \frac{b - a}{2}(f_0 + f_1)$ \\
			Für ein Intervall: 
			\[
			\int_{1}^{e}\mathrm{log}(x)dx = \dfrac{e - 1}{2}(\mathrm{log}(1) + \mathrm{log}(e)) = 0.8591409142295225
			\]
			Für mehrere Intervalle:\\
			\begin{tabular}{|c|c|}
				\hline 
				Intervalle & Ergebnis \\ 
				\hline 
				2 & 0.9623362015498855 \\ 
				\hline 
				4 & 0.9903650088127982 \\ 
				\hline 
				8 & 0.9975754238329705\\ 
				\hline 
				16 & 0.999392820210616\\ 
				\hline 
			\end{tabular} \\
			\item Simpson-Regel: $\int_{a}^{b} f(x) dx = \frac{b - a}{6}(f_0 + 4f_1(\frac{a + b}{2}) + f_2)$\\
			Für ein Intervall:
			\[
			\int_{1}^{e}\mathrm{log}(x)dx = \dfrac{e - 1}{6}(\mathrm{log}(1) + 4\mathrm{log}\Big(\frac{1+e}{2}\Big) + \mathrm{log}(e)) = 0.996734630656673
			\]
			Für mehrere Intervalle:\\
			\begin{tabular}{|c|c|}
				\hline 
				Intervalle & Ergebnis \\ 
				\hline 
				2 & 0.9997079445671028 \\ 
				\hline 
				4 & 0.9999788955063613 \\ 
				\hline 
				8 & 0.9999986190031646\\ 
				\hline 
				16 & 0.9999999126024922\\ 
				\hline 
			\end{tabular} \\
			\item exaktes Ergebnis\\
			\[
			\int^e_1 \mathrm{log}(x)dx = \int^e_1 1 \cdot \mathrm{log}(x)dx = x \cdot \mathrm{log}(x) - \int^e_1 x \cdot \frac{1}{x}dx = \big[x \cdot \mathrm{log}(x) - x + c \big]^e_1
			\]
			\[
			\Rightarrow \ (e \cdot \mathrm{log}(e) - e) - (1 \cdot \mathrm{log}(1) - 1) = 1
			\]
			\item 
			zugrunde liegender Quellcode (Java)
			\begin{lstlisting}
double a = 1;
double b = Math.E;
double steps = 0;
double result = 0;

System.out.println("Funktion log(x): ");
System.out.println("***Trapezregel***");
for (int i = 1; i <= 16; i = i * 2) {
	steps = (b - a) / i;
	for (int s = 1; s <= i; s++) {
		result += (steps / 2) * (Math.log(a) + Math.log(a + steps));
		a += steps;
	}
	System.out.println("Intervalle: " + i + " Ergebnis: " + result);
	a = 1;
	result = 0;
}
			\end{lstlisting}
			\newpage
			\begin{lstlisting}
System.out.println("\n***Simpson-Regel***");
for (int i = 1; i <= 16; i = i * 2) {
	steps = (b - a) / i;
	for (int s = 1; s <= i; s++) {
		result += (steps / 6) * (Math.log(a) + Math.log(a + steps)
			+ 4 * Math.log((steps + a + a) / 2));
		a += steps;
	}
   	System.out.println("Intervalle: " + i + " Ergebnis: " + result);
	a = 1;
	result = 0;
}
			\end{lstlisting}
		\end{itemize}
		\item 
		\begin{itemize}
			\item 
			Trapezregel: $\int_{a}^{b} f(x) dx = \frac{b - a}{2}(f_0 + f_1)$ \\
			Für
			\begin{align*}
			8 \text{ Intervalle } &  : \ 0.3359375\\
			16 \text{ Intervalle } & : \ 0.333984375
			\end{align*}
			\item
			exaktes Ergebnis:
			\[
			\int^1_0 x^2 dx = \Big [\frac{x^3}{3} + c \Big]^1_0 \quad \Rightarrow \quad \frac{1^3}{3} - \frac{0^3}{3} = \frac{1}{3} \approx 0.3333333333
			\]
			\item 
			zugrunde liegender Quellcode (Java)
			\begin{lstlisting}
double a = 0;
double b = 1;
double steps = 0;
double result = 0;
System.out.println("\n\nFunktion x^2: ");
System.out.println("***Trapezregel***");
for (int i = 8; i <= 16; i = i * 2) {
	steps = (b - a) / i;
	for (int s = 1; s <= i; s++) {
		result += (steps / 2) * ((a * a) + ((a + steps) * (a + steps)));
		a += steps;
	}
	System.out.println("Intervalle: " + i + " Ergebnis: " + result);
	a = 0;
	result = 0;
}			
			\end{lstlisting}
		\end{itemize}
	\end{enumerate}
\end{document}
