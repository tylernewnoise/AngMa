\documentclass[a4paper,10pt]{article}
\usepackage[ngerman]{babel}		%dt. Übersetzung und Umlaute
\usepackage[utf8]{inputenc}		%Umlaute direkt eingeben
\usepackage{mathtools}			%Mathekrams
\usepackage{paralist}			%bessere Listen
\usepackage{amssymb}			%Mathesymbole
\usepackage{amsthm}				%typesetting theorems
\usepackage{colortbl}

\usepackage{fancyhdr} 			%Headerstyles
\usepackage[margin=2.0cm,headheight=40pt,top=3cm]{geometry}
\pagestyle{fancy}

\renewcommand{\headrulewidth}{0.4pt}
\renewcommand{\footrulewidth}{0.4pt}
\lhead{Blatt 05 - Teil 2}
\rhead{Angewandte Mathematik}
\cfoot{}
\rfoot{\thepage}
\begin{document}
	\parindent0pt
	\textbf{Aufgabe 03}\\
	\begin{compactenum}[a)]
		\item 
		\begin{compactenum}
			\item [a1)]
			Newton-Verfahren für $f_2(x) = \sqrt{x + 1} + \frac{\mathrm{sin}(x)}{10} - 2$ mit $x_0 = 8$:\\\\
			$f'_(x) = \frac{1}{2\sqrt{x + 1}} + \frac{\mathrm{cos}(x)}{10}$\\\\\\
			\setlength{\extrarowheight}{12pt}
			\begin{tabular}{|c|c|}
				\hline
				$x_1 = x_0 - \dfrac{f(x_0)}{f'(x_0)} = 8 - \dfrac{3 + \frac{\mathrm{sin(8)}}{10} - 2}{\frac{1}{6} + \frac{\mathrm{8}}{10}}$ & 0,775703851736507 \\
				$x_2 = x_1 - \dfrac{f(x_1)}{f'(x_1)}$ & 2,11338187957056 \\ 
				$x_3 = x_2 - \dfrac{f(x_2)}{f'(x_2)}$ & 2,76017545567755\\
				$x_4 = x_3 - \dfrac{f(x_3)}{f'(x_3)}$ & 2,90353368300056\\
				$x_5 = x_4 - \dfrac{f(x_4)}{f'(x_4)}$ & 2,90790908441486\\
				\hline
			\end{tabular}
			\setlength{\extrarowheight}{0pt}
			\ \\\\
			$x$ nach 5 Iterationen: 2,90790908441486.\\
			\item [a2)]
			Modifiziertes Newton-Verfahren für $f_2(x) = \sqrt{x + 1} + \frac{\mathrm{sin}(x)}{10} - 2$ mit $x_0 = 8$:\\\\
			$f'_(x) = \frac{1}{2\sqrt{x + 1}} + \frac{\mathrm{cos}(x)}{10}$\\\\\\
			\setlength{\extrarowheight}{12pt}
			\begin{tabular}{|c|c|}
				\hline
				$x_1 = x_0 - \dfrac{f(x_0)}{f'(x_0)} = 8 - \dfrac{3 + \frac{\mathrm{sin(8)}}{10} - 2}{\frac{1}{6} + \frac{\mathrm{8}}{10}}$ & 0,775703851736507 \\
				$x_2 = x_1 - \dfrac{f(x_1)}{f'(x_0)}$ & 0,775703851736507 \\
				$x_3 = x_2 - \dfrac{f(x_2)}{f'(x_0)}$ & 4,70310225854163 \\
				$x_4 = x_3 - \dfrac{f(x_3)}{f'(x_0)}$ & 2,80902181167063 \\
				$x_5 = x_4 - \dfrac{f(x_4)}{f'(x_0)}$ & 2,90781838247131 \\
				\hline
			\end{tabular}
			\setlength{\extrarowheight}{0pt}
			\ \\\\
			$x$ nach 5 Iterationen: 2,90781838247131.\\\\
		\end{compactenum}
		\item TODO
	\end{compactenum}
	\newpage
	\textbf{Aufgabe 4}
	\begin{compactenum}[a)]
		\item 
		Hornerschema für $f(x) = -x^3 + 3x^2 - x - 1$:
		\begin{align*}
			f(x) & = -x^3 + 3x^2 - x - 1\\
			& = (-x^2 + 3x -1)x - 1\\
			& = (-x +3)x - 1)x - 1\\
			& \overset{x = 2,5}{=} (-2,5 + 3) \cdot 2,5 - 1) \cdot 2,5 - 1)\\
			& = (1,25 - 1) \cdot 2,5 - 1)\\
			& = 0,25 \cdot 2,5 - 1\\
			& = \underline{\underline{-0,375}}
		\end{align*}\\
		Hornerschema für $f'(x) = -3x^2 + 6x - 1$:\\
		\begin{align*}
			f'(x) & = -3x^2 + 6x - 1\\
			& = (-3x + 6)x - 1\\
			& \overset{x = 25}{=} (-7,5 + 6) \cdot 2,5 -1\\
			& = -1,5 \cdot 2,5 - 1\\
			& = -3,75 - 1\\
			& = \underline{\underline{-4,75}}
		\end{align*}
		\ \\
		\item 
		\begin{compactenum}
			\item [b1)]
			Intervallschachtelung für $f(x) = -x^3 + 3x^2 - x - 1$ im Intervall [2, 2,5]:\\
			\begin{tabular}{|crrrr|}
				\hline
				$i$ & $a$ & $b$ & $x_i$ & $f(x_i)$\\
				0 & 2 & 2,5 & 2,25 & 0,546875 \\
				1 & 2,25 & 2,5 & 2,375 & 0,150390625 \\
				2 & 2,375 & 2,5 & 2,4375 & -0,095458984375 \\
				3 & 2,375 & 2,4375 & 2,40625 & 0,031585693359375 \\
				4 & 2,40625 & 2,4375 & 2,421875 & -0,030895233154297 \\
				5 & 2,40625 & 2,421875 & 2,4140625 & 0,000604152679443 \\
				\hline
			\end{tabular}
			\\\\
			$x$ nach 5 Iterationen: 2,4140625.\\
			\item [b2)]
			Newtonverfahren für $f(x) = -x^3 + 3x^2 - x - 1$ mit $x_0 = 2,5$:\\
			\begin{tabular}{|ccc|}
				\hline
				0 & 2,5 & -0,375 \\
				1 & 2,42105263157895 & -0,027555037177431 \\
				2 & 2,41426261900485 & -0,000196236737278 \\
				\hline
			\end{tabular}
			\\\\\
			$x$ nach 2 Iterationen: 2,41426261900485.
		\end{compactenum}
	\end{compactenum}
	

\end{document}
