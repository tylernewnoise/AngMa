\documentclass[a4paper,10pt]{article}
\usepackage[ngerman]{babel}		%dt. Übersetzung und Umlaute
\usepackage[utf8]{inputenc}		%Umlaute direkt eingeben
\usepackage{mathtools}			%Mathekrams
\usepackage{paralist}			%bessere Listen
\usepackage{amssymb}			%Mathesymbole
\usepackage{amsthm}				%typesetting theorems

\usepackage{fancyhdr} 			%Headerstyles
\usepackage[margin=2.0cm,headheight=40pt,top=3cm]{geometry}
\pagestyle{fancy}

\renewcommand{\headrulewidth}{0.4pt}
\renewcommand{\footrulewidth}{0.4pt}
\lhead{Blatt 05 - Teil 1}
\rhead{Angewandte Mathematik}
\cfoot{}
\rfoot{\thepage}
\begin{document}
	\parindent0pt
	\textbf{Aufgabe 01}\\
	TODO
	\newpage
	\textbf{Aufgabe 02}
	\begin{enumerate}[a)]
		\item Fixpunktiteration für $f_1(x) = \frac{1}{2} \mathrm{cos}(x) - x$\\
		Voraussetzung prüfen: TODO\\\\\\
		\begin{tabular}{|c|c|c|c|}
			\hline 
			$i$& $x_{i-1}$ & $f(x_{i-1})$ & $f(x_{i-1}) + x_{i-1}$ \\ 
			\hline 
			0 & 0 & 0,5 & 0,5 \\ 
			\hline 
			1 & 0,5 & -0,061208719054814 & 0,438791280945186 \\
			\hline 
			2 &  0,438791280945186 & 0,013841640715023 & 0,45263292166021  \\
			\hline 
			3 & 0,45263292166021 & -0,00298354544655 & 0,44964937621366  \\
			\hline 
			4 & 0,44964937621366 & 0,00065040191779	& 0,45029977813145 \\
			\hline 
			5 & 0,44964937621366 & 0,00065040191779 & 0,45029977813145 \\
			\hline 
		\end{tabular} \\\\
		$x_5 = 0,45029977813145$
	
	
	
	
	
	
	
		\item 
		\item 
	\end{enumerate}
	

\end{document}
