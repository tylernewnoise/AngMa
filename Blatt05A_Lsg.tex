\documentclass[a4paper,10pt]{article}
\usepackage[ngerman]{babel}		%dt. Übersetzung und Umlaute
\usepackage[utf8]{inputenc}		%Umlaute direkt eingeben
\usepackage{mathtools}			%Mathekrams
\usepackage{paralist}			%bessere Listen
\usepackage{amssymb}			%Mathesymbole
\usepackage{amsthm}				%typesetting theorems

\usepackage{fancyhdr} 			%Headerstyles
\usepackage[margin=2.0cm,headheight=40pt,top=3cm]{geometry}
\pagestyle{fancy}

\renewcommand{\headrulewidth}{0.4pt}
\renewcommand{\footrulewidth}{0.4pt}
\lhead{Blatt 05 - Teil 1}
\rhead{Angewandte Mathematik}
\cfoot{}
\rfoot{\thepage}
\begin{document}
	\parindent0pt
	\textbf{Aufgabe 01}\\
	\begin{compactenum}[a)]
		\item TODO
		\item $f$ und $g$ sind für Single Precision numerisch stabil, für Double Precision nicht. Dies sieht man daran, dass sich die Ergebnisse von $f$ und $g$ in Double Precision unterscheiden.\\
		Da wir uns Mathematica nicht leisten können, haben wir die Funktionen selbst implementiert. Die Ergebnisse sind:\\\\
		\begin{tabular}{|c|r|r|r|r|}
			\hline 
			$i$ &	$f(x)$ double	&	$f(x)$ single &	$g(x)$ double	&	$g(x)$ single	\\
			\hline
			1	&	 0.1180339887498949	&	 0.118034005	&	0.11803398874989485	&	 0.118034005	\\
			2	&	0.030776406404415146	&	0.030776381	&	0.030776406404415133	&	 0.030776381	\\
			3	&	0.0077822185373186414	&	0.007782221	&	0.0077822185373187065	&	 0.007782221	\\
			4	&	0.001951221367587408	&	0.0019512177	&	0.0019512213675873353	&	 0.0019512177	\\
			5	&	4.88162098882583E-4	&	 4.8816204E-4	&	4.881620988826073E-4	&	 4.8816204E-4	\\
			6	&	1.2206286282867573E-4	&	1.2207031E-4	&	1.2206286282875901E-4	&	 1.2207031E-4	\\
			7	&	3.0517112477923547E-5	&	3.0517578E-5	&	3.0517112477923005E-5	&	 3.0517578E-5	\\
			8	&	7.629365427641588E-6	&	7.6293945E-6	&	7.629365427641587E-6	&	 7.6293945E-6	\\
			9	&	1.9073468138230965E-6	&	1.9073486E-6	&	1.907346813826566E-6	&	 1.9073486E-6	\\
			10	&	4.768370445162873E-7	&	4.7683716E-7	&	4.768370445163415E-7	&	 4.7683716E-7	\\
			11	&	1.1920928244535389E-7	&	1.1920929E-7	&	1.1920928244535474E-7	&	 1.1920929E-7	\\
			12	&	2.9802321943606103E-8	&	0.0	&	2.9802321943606116E-8	&	 0.0	\\
			13	&	7.450580596923828E-9	&	0.0	&	7.4505805691682525E-9	&	 0.0	\\
			14	&	1.862645149230957E-9	&	0.0	&	1.8626451474962336E-9	&	 0.0	\\
			15	&	4.6566128730773926E-10	&	0.0	&	4.6566128719931904E-10	&	 0.0	\\
			16	&	1.1641532182693481E-10	&	0.0	&	1.1641532182015855E-10	&	 0.0	\\
			17	&	2.9103830456733704E-11	&	0.0	&	2.9103830456310187E-11	&	 0.0	\\
			18	&	7.275957614183426E-12	&	0.0	&	7.275957614156956E-12	&	 0.0	\\
			19	&	1.8189894035458565E-12	&	0.0	&	1.8189894035442021E-12	&	 0.0	\\
			20	&	4.547473508864641E-13	&	0.0	&	4.547473508863607E-13	&	 0.0	\\
			21	&	1.1368683772161603E-13	&	0.0	&	1.1368683772160957E-13	&	 0.0	\\
			22	&	2.8421709430404007E-14	&	0.0	&	2.8421709430403604E-14	&	 0.0	\\
			23	&	7.105427357601002E-15	&	0.0	&	7.105427357600977E-15	&	 0.0	\\
			24	&	1.7763568394002505E-15	&	0.0	&	1.7763568394002489E-15	&	 0.0	\\
			25	&	4.440892098500626E-16	&	0.0	&	4.440892098500625E-16	&	 0.0	\\
			\hline 
		\end{tabular} \\\\
	Das den Ergebnissen zugrunde liegende Programm (Java-Implementation):
	\begin{verbatim}
		public class Main {
		        public static void main(String[] args) {
		                double f_double_prec = 0;
		                float f_single_prec = 0;
		                double x = 0;

		                System.out.println("erste Funktion f(x)");
		                for (int i = 1; i <= 25; i++) {
		                        x = 1 / (Math.pow(2, i));
		                        f_double_prec = Math.sqrt(x * x + 1) - 1;
		                        f_single_prec = (float) Math.sqrt(x * x + 1) - 1;
		                        System.out.println("double: " + f_double_prec + "\t\t\t  float: " 
		                        + f_single_prec);
		                }

		                System.out.println();
		                System.out.println("zweite Funktion g(x)");
		                for (int i = 1; i <= 25; i++) {
		                        x = 1 / (Math.pow(2, i));
		                        f_double_prec = (x * x) / (Math.sqrt(x * x + 1) + 1);
		                        f_single_prec = (float) Math.sqrt(x * x + 1) - 1;
		                        System.out.println("double: " + f_double_prec + "\t\t\t  float: " 
		                        + f_single_prec);
		                }
		        }
		}
	\end{verbatim}
	
	\end{compactenum}

	\newpage
	\textbf{Aufgabe 02}
	\begin{enumerate}[a)]
		\item Fixpunktiteration für $f_1(x) = \frac{1}{2} \mathrm{cos}(x) - x$\\\\
		Bedingung: f.a. $x \in M \quad \mathrm{sup}\Vert I + J(x) \Vert < 1, x \in M$\\
		$J(x) = \Big (\dfrac{-\mathrm{sin}(x)}{2} - 1 \Big)$\\
		$I + J(x) = \Big (\dfrac{-\mathrm{sin}(x)}{2} - 1 + 1 \Big) = \Big ( \dfrac{-\mathrm{sin}(x)}{2} \Big)$\\
		$\mathrm{sup} \Vert \dfrac{-\mathrm{sin}(x)}{2} \Vert = 0,5 < 1$\\

		\begin{tabular}{|c|c|c|c|}
			\hline 
			$i$& $x_{i-1}$ & $f(x_{i-1})$ & $f(x_{i-1}) + x_{i-1}$ \\ 
			\hline 
			0 & 0 & 0,5 & 0,5 \\ 
			\hline 
			1 & 0,5 & -0,061208719054814 & 0,438791280945186 \\
			\hline 
			2 &  0,438791280945186 & 0,013841640715023 & 0,45263292166021  \\
			\hline 
			3 & 0,45263292166021 & -0,00298354544655 & 0,44964937621366  \\
			\hline 
			4 & 0,44964937621366 & 0,00065040191779	& 0,45029977813145 \\
			\hline 
			5 & 0,44964937621366 & 0,00065040191779 & 0,45029977813145 \\
			\hline 
		\end{tabular} \\\\
		$\underline{\underline{x = 0,45029977813145}}$
		\item Jacobi-Verfahren für $f(x,y) = 
		\begin{pmatrix}
		2 & 1\\
		1 & 2
		\end{pmatrix}
		\begin{pmatrix}
		x\\
		y
		\end{pmatrix}
		-
		\begin{pmatrix}
		1\\
		0
		\end{pmatrix}$\\
		$A = 
		\begin{pmatrix}
		2 & 1\\
		1 & 2
		\end{pmatrix}
		\quad
		b =
		\begin{pmatrix}
		1\\
		0
		\end{pmatrix}$\\\\
		$x_{m + 1} = (I - C \cdot A) \cdot x_m + C \cdot b$\\
		\[
		C = 
		\begin{pmatrix}
		\frac{1}{2} & 0 \\[2pt]
		0 & \frac{1}{2}
		\end{pmatrix}
		\quad
		C \cdot A =
		\begin{pmatrix}
		\frac{1}{2} & 0\\[2pt]
		0 & \frac{1}{2}
		\end{pmatrix}
		\cdot
		\begin{pmatrix}
		2 & 1\\
		1 & 2
		\end{pmatrix}
		=
		\begin{pmatrix}
		1 & \frac{1}{2} \\[2pt]
		\frac{1}{2} & 1
		\end{pmatrix}
		\]
		\[
		I - C \cdot A =
		\begin{pmatrix}
		1 & 0\\
		0 & 1
		\end{pmatrix}
		-
		\begin{pmatrix}
		1 & \frac{1}{2} \\[2pt]
		\frac{1}{2} & 1
		\end{pmatrix}
		=
		\begin{pmatrix}
		0 & \frac{1}{2} \\[2pt]
		\frac{1}{2} & 0
		\end{pmatrix}
		\]
		\[
		C \cdot b = 
		\begin{pmatrix}
		\frac{1}{2} & 0\\[2pt]
		0 & \frac{1}{2}
		\end{pmatrix}
		\cdot
		\begin{pmatrix}
		1\\
		0
		\end{pmatrix}
		=
		\begin{pmatrix}
		\frac{1}{2}\\[2pt]
		\frac{1}{2}
		\end{pmatrix}
		\]
		\begin{align*}
		x_1 = & 
		\begin{pmatrix}
		0 & \frac{1}{2}\\[2pt]
		\frac{1}{2} & 0
		\end{pmatrix} 
		\cdot
		\begin{pmatrix}
		0\\
		0
		\end{pmatrix}
		+
		\begin{pmatrix}
		\frac{1}{2}\\[2pt]
		\frac{1}{2}
		\end{pmatrix}
		=
		\begin{pmatrix}
		\frac{1}{2}\\[2pt]
		\frac{1}{2}
		\end{pmatrix}\\
		x_2 = &
		\begin{pmatrix}
		0 & \frac{1}{2}\\[2pt]
		\frac{1}{2} & 0
		\end{pmatrix} 
		\cdot
		\begin{pmatrix}
		\frac{1}{2}\\[2pt]
		\frac{1}{2}
		\end{pmatrix}
		+
		\begin{pmatrix}
		\frac{1}{2}\\[2pt]
		\frac{1}{2}
		\end{pmatrix}
		=
		\begin{pmatrix}
		\frac{1}{4}\\[2pt]
		\frac{1}{4}
		\end{pmatrix}
		+
		\begin{pmatrix}
		\frac{1}{2}\\[2pt]
		\frac{1}{2}
		\end{pmatrix}
		=
		\begin{pmatrix}
		\frac{3}{4}\\[2pt]
		\frac{3}{4}
		\end{pmatrix}\\
		x_3 = &
		\begin{pmatrix}
		0 & \frac{1}{2}\\[2pt]
		\frac{1}{2} & 0
		\end{pmatrix} 
		\cdot
		\begin{pmatrix}
		\frac{3}{4}\\[2pt]
		\frac{3}{4}
		\end{pmatrix}
		+
		\begin{pmatrix}
		\frac{1}{2}\\[2pt]
		\frac{1}{2}
		\end{pmatrix}
		=
		\begin{pmatrix}
		\frac{3}{8}\\[2pt]
		\frac{3}{8}
		\end{pmatrix}
		+
		\begin{pmatrix}
		\frac{1}{2}\\[2pt]
		\frac{1}{2}
		\end{pmatrix}
		=
		\begin{pmatrix}
		\frac{7}{8}\\[2pt]
		\frac{7}{8}
		\end{pmatrix}\\
		x_4 = &
		\begin{pmatrix}
		0 & \frac{1}{2}\\[2pt]
		\frac{1}{2} & 0
		\end{pmatrix} 
		\cdot
		\begin{pmatrix}
		\frac{7}{8}\\[2pt]
		\frac{7}{8}
		\end{pmatrix}
		+
		\begin{pmatrix}
		\frac{1}{2}\\[2pt]
		\frac{1}{2}
		\end{pmatrix}
		=
		\begin{pmatrix}
		\frac{7}{16}\\[2pt]
		\frac{7}{16}
		\end{pmatrix}
		+
		\begin{pmatrix}
		\frac{1}{2}\\[2pt]
		\frac{1}{2}
		\end{pmatrix}
		=
		\begin{pmatrix}
		\frac{15}{16}\\[2pt]
		\frac{15}{16}
		\end{pmatrix}\\
		x_5 = &
		\begin{pmatrix}
		0 & \frac{1}{2}\\[2pt]
		\frac{1}{2} & 0
		\end{pmatrix} 
		\cdot
		\begin{pmatrix}
		\frac{15}{16}\\[2pt]
		\frac{15}{16}
		\end{pmatrix}
		+
		\begin{pmatrix}
		\frac{1}{2}\\[2pt]
		\frac{1}{2}
		\end{pmatrix}
		=
		\begin{pmatrix}
		\frac{15}{32}\\[2pt]
		\frac{15}{32}
		\end{pmatrix}
		+
		\begin{pmatrix}
		\frac{1}{2}\\[2pt]
		\frac{1}{2}
		\end{pmatrix}
		=
		\begin{pmatrix}
		\frac{31}{32}\\[2pt]
		\frac{31}{32}
		\end{pmatrix}
		\end{align*}
		$\underline{\underline{x =
		\begin{pmatrix}
		\frac{31}{32}\\[2pt]
		\frac{31}{32}
		\end{pmatrix}}}$
		\newpage
		\item 
		Modifizierte einfache Iteration für $f(x) = \sqrt{x} + \frac{\mathrm{cos}(x)}{15} - 2$\\
		$f'(x) = \frac{1}{2 \sqrt{x}} - \frac{\mathrm{sin}(x)}{15}$\\
		Schätze Grenzen von $f'(x)$:
		\[
		\frac{1}{10} = \frac{1}{6} - \frac{1}{15} \leq f'(x) = \frac{1}{2\sqrt{x}} - \frac{\mathrm{sind}(x)}{15} \leq \frac{1}{2} - \frac{-1}{15} =  \frac{17}{30}
		\]
		Sei $\beta = -1$\\
		\begin{align*}
			\Rightarrow \text{linke Grenze: } & \vert 1 + (-1) \cdot \frac{1}{10} \vert = \frac{9}{10} < 1 \quad \checkmark\\
			\Rightarrow \text{rechte Grenze: } & \vert 1 + (-1) \cdot \frac{17}{30} \vert = \frac{13}{30} < 1 \quad \checkmark\\
			& \Rightarrow \vert 1 + (-1) \cdot f'(x) \vert < 1 
		\end{align*}
		\[
		x_n = x_{n - 1} + \beta + f(x_{n -1}) \quad \quad n \in \mathbb{N}
		\]
		\begin{center}
			\begin{tabular}{|c|c|}
			\hline 
			$x_1 = x_0 + (-1) \cdot f(x_0) = $ & 8,06074201745898 \\
			\hline
			$x_2 = x_1 + (-1) \cdot f(x_1) = $ & 7,23528344932496 \\
			\hline 
			$x_3 = x_2 + (-1) \cdot f(x_2) = $ & 6,50677021889058 \\
			\hline
			$x_4 = x_3 + (-1) \cdot f(x_3) = $ & 5,89092580207711 \\
			\hline
			$x_5 = x_4 + (-1) \cdot f(x_4) = $ & 5,40219967715733 \\
			\hline
			$x_6 = x_5 + (-1) \cdot f(x_5) = $ & 5,03551034435828 \\
 			\hline 
			\end{tabular} 
		\end{center}
		TODO: Zeigen, dass das Iterationsverfahren den Definitionsbereich nicht verlässt
	\end{enumerate}
	

\end{document}
