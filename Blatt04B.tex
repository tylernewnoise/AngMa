\documentclass[a4paper,10pt]{article}
\usepackage[ngerman]{babel}		%dt. Übersetzung und Umlaute
\usepackage[utf8]{inputenc}		%Umlaute direkt eingeben
\usepackage{mathtools}			%Mathekrams
\usepackage{paralist}			%bessere Listen
\usepackage{amssymb}			%Mathesymbole
\usepackage{amsthm}				%typesetting theorems (Text über = u.ä.)
\usepackage{fancyhdr} 			%Headerstyles
\usepackage{colortbl}			%Farben in Tabellen
\usepackage{hhline}				%Rahmen in farbigen Cellen
\usepackage[margin=2.0cm,headheight=40pt,top=3cm]{geometry}
\pagestyle{fancy}

\renewcommand{\headrulewidth}{0.4pt}
\renewcommand{\footrulewidth}{0.4pt}
\lhead{Blatt 04 - Teil 2}
\rhead{Angewandte Mathematik}
\cfoot{}
\rfoot{\thepage}
\begin{document}
	\parindent0pt
	\textbf{Aufgabe 03}\\
	
	Beweis durch Kontraposition: Angenommen die drei Aussagen stimmen. Wir finden ein Gegenbeispiel für alle drei Aussagen:\\
	
	\begin{compactenum} [a)]
		\item Sei $ a := (-1)^0\cdot2^{-3}\cdot0,10\underbrace{11...11}_{50 mal} \qquad b := (-1)^0\cdot2^{-6}\cdot0,111101 \Longrightarrow a, b \in \mathbb{G}$\\
		$ a+b = (-1)^0\cdot2^{-6}\cdot0,00010\underbrace{11...11}_{50 mal} + (-1)^0\cdot2^{-6}\cdot0,111101 \\
		= (-1)^0\cdot2^{-7}\cdot0,1000010\underbrace{11..11}_{49 mal} \not\in \mathbb{G}$
		\item  Sei $ a:= (-1)^0\cdot2^{-52}\cdot0,\underbrace{11..11}_{52 mal} \qquad b:= (-1)^0\cdot2^0\cdot0,1$\\
		\begin{tabbing}
			$ a\boxplus b$ \=$= rd(a+b)$ \\
			\>$ =rd((-1)^0\cdot2^{-52}\cdot0,\underbrace{11..11}_{52 mal}+(-1)^0\cdot2^0\cdot0,1) $ \\
			\>$ =rd((-1)^0\cdot2^{-52}\cdot0,\underbrace{11..11}_{52 mal}+(-1)^0\cdot2^{-52}\cdot0,\underbrace{00..00}_{52 mal}1) $ \\
			\>$ =rd((-1)^0\cdot2^{-52}\cdot0,\underbrace{11..11}_{53 mal})$\\
			\>$ =(-1)^0\cdot2^{-52}\cdot0,\underbrace{11..11}_{52 mal}$\\
			\>$ =a $
		\end{tabbing}
		\item Sei $ a:= (-1)^0\cdot2^{-52}\cdot0,\underbrace{11..11}_{52 mal} \qquad b:= (-1)^0\cdot2^0\cdot0,1 \qquad c:= b $\\
		$ (a\boxplus b) \boxplus c \overset{\text{siehe Aufgabe b)}}{=} a \boxplus c \overset{\text{weil b $\equiv$ c}}{=} a$\\
		\begin{tabbing}
			$ a\boxplus (b\boxplus c) $ \=$ = a \boxplus (rd((-1)^0\cdot2^0\cdot0,1+(-1)^0\cdot2^0\cdot0,1))$\\
			\> $ =a \boxplus (rd((-1)^0\cdot2^{-1}\cdot0,1)) $\\
			\> $ =a \boxplus (-1)^0\cdot2^{-1}\cdot0,1 $\\
			\> $ =rd((-1)^0\cdot2^{-52}\cdot0,\underbrace{11..11}_{52 mal}+(-1)^0\cdot2^{-1}\cdot0,1) $\\
			\> $ =rd((-1)^0\cdot2^{-52}\cdot0,\underbrace{11..11}_{52 mal}+(-1)^0\cdot2^{-52}\cdot0,\underbrace{00..00}_{51 mal}1 $\\
			\> $ =rd((-1)^0\cdot2^{-52}\cdot1,0) $ \\
			\> $ =rd((-1)^0\cdot2^{-53}\cdot0,1) $\\
			\> $ \neq a $
		\end{tabbing}
	\end{compactenum}
	\newpage
	\textbf{Aufgabe 04}\\
	\begin{enumerate}[a)]
		\item
			\begin{enumerate}
				\item[a1)]
				Jacobi-Matrix:
				\[
				J = 
				\begin{pmatrix}
					\dfrac{1}{y} \quad -\dfrac{x}{y^2}
				\end{pmatrix}
				\]
				Absolute komponentenweise Kondition:
				\[
				\kappa^{abs}_{1,1} = \Big\vert \dfrac{1}{y} \Big\vert \quad\quad \kappa^{abs}_{1,2} = \Big\vert -\dfrac{x}{y^2} \Big \vert
				\]
				Relative komponentenweise Kondition:
				\[
				\kappa^{rel}_{1,1} = \Big\vert \dfrac{1}{y}\Big\vert \cdot \dfrac{\vert x \vert}{\Big \vert \dfrac{x}{y} \Big\vert} \quad\quad \kappa^{rel}_{1,2} = \Big\vert -\dfrac{x}{y^2}\Big\vert \cdot \dfrac{\vert y \vert}{\Big \vert \dfrac{x}{y} \Big\vert}
				\]
				Absolute normweise Kondition:
				\[
				\kappa_{abs}(x,y) = \Vert J \Vert = max\{\Big\vert \dfrac{1}{y} \Big\vert + \Big\vert -\dfrac{x}{y^2} \Big \vert\}
				\]
				Relative normweise Kondition:
				\[
				\kappa_{rel}(x,y) = \dfrac{\vert x \vert + \vert y \vert}{\Big\vert \dfrac{1}{y} \Big\vert + \Big\vert \dfrac{x}{y^2} \Big \vert}
				\]
				\item[a2)]
				Jacobi-Matrix:
				\[
				J = 
				\begin{pmatrix}
					y \cdot cos(x \cdot y) \quad x \cdot cos(x \cdot y)
				\end{pmatrix}
				\]
				Absolute komponentenweise Kondition:
				\[
				\kappa^{abs}_{1,1} = \Big\vert y \cdot cos(x \cdot y) \Big\vert \quad\quad \kappa^{abs}_{1,2} = \Big\vert x \cdot cos(x \cdot y) \Big \vert
				\]
				Relative komponentenweise Kondition:
				\[
				\kappa^{rel}_{1,1} = \Big\vert y \cdot cos(x \cdot y) \Big\vert \cdot \dfrac{\vert x \vert}{\Big \vert sin(x \cdot y) \Big\vert} \quad\quad \kappa^{rel}_{1,2} = \Big\vert x \cdot cos(x \cdot y) \Big\vert \cdot \dfrac{\vert y \vert}{\Big \vert sin(x \cdot y) \Big\vert}
				\]
				Absolute normweise Kondition:
				\[
				\kappa_{abs}(x,y) = \Vert J \Vert = max\{\vert y \cdot cos(x \cdot y)\vert + \vert x \cdot cos(x \cdot y) \vert\}
				\]
				Relative normweise Kondition:
				\[
				\kappa_{rel}(x,y) = \dfrac{\vert x \vert + \vert y \vert}{\vert y \cdot cos(x \cdot y) \vert + \vert x \cdot cos(x \cdot y \vert}
				\]
			\end{enumerate}
				\item
			\begin{enumerate}
				\item
				\item
				\item
			\end{enumerate}


	\end{enumerate}
	
	

\end{document}
