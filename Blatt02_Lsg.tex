\documentclass[a4paper,10pt]{article}
\usepackage[ngerman]{babel}		%dt. Übersetzung und Umlaute
\usepackage[utf8]{inputenc}		%Umlaute direkt eingeben
\usepackage{mathtools}			%Mathekrams
\usepackage{paralist}			%bessere Listen
\usepackage{amssymb}			%Mathesymbole
\usepackage{amsthm}				%typesetting theorems (Text über = u.ä.)
\usepackage{fancyhdr} 			%Headerstyles
\usepackage{colortbl}			%Farben in Tabellen
\usepackage{hhline}				%Rahmen in farbigen Cellen
\usepackage[margin=2.0cm,headheight=40pt,top=3cm]{geometry}
\pagestyle{fancy}

\renewcommand{\headrulewidth}{0.4pt}
\renewcommand{\footrulewidth}{0.4pt}
\lhead{Blatt 02}
\rhead{Angewandte Mathematik}
\cfoot{}
\rfoot{\thepage}
\begin{document}
	\parindent0pt
	\textbf{Aufgabe 03}\\
	Zusammenfassung:
	\begin{itemize}
		\item Produkt 1: $x_1$\\
		Produkt 2: $x_2$
		\item Gewinn soll maximiert werden: $2000x_1 + 3000x_3$
		\item Maschinen laufen 12h und 16h und ben"otigen 2h und 2h bzw. 4h zur Herstellung eines Produkts:
		\[
		\begin{matrix}
			2x_1 + 2x_2 \leq 12\\
			4x_1 + 0x_2 \leq 16
		\end{matrix}
		\]
		\item 20 Einheiten an Material sind vorhanden, wobei $x_1$ 2 Einheiten und $x_2$ 4 Einheiten an den Maschinen ben"otigt:
		\[
		\begin{matrix}
		2x_1 + 4x_2 \leq 20\\
		2x_1 + 0x_2 \leq 20
		\end{matrix}
		\]
	\end{itemize}
	Daraus ergibt sich:
	\[
	(P) \text{ max} \{\ 2000x_1 + 3000x_2 \mid \begin{matrix}
	2x_1 + 2x_2 \leq 12\\
	4x_1 + 0x_2 \leq 16\\
	2x_1 + 4x_2 \leq 20\\
	2x_1 + 0x_2 \leq 20\\
	x_1, x_2 \geq 0
	\end{matrix}
	\
	\}
	\]
	Nach Hinzuf"ugen von Schlupfvariablen erhalten wir die folgende Matrix:
	\[
	A:=
	\begin{pmatrix}
	2 & 2 & 1 & 0 & 0 & 0\\
	4 & 0 & 0 & 1 & 0 & 0\\
	2 & 4 & 0 & 0 & 1 & 0\\
	2 & 0 & 0 & 0 & 0 & 1
	\end{pmatrix} \quad b:= 
	\begin{pmatrix}
	12\\
	16\\
	20\\
	20
	\end{pmatrix}
	\]
	Daraus ergeben sich die folgenden Simplextableaus:\\
	\[
	\setlength{\extrarowheight}{2pt}
	\begin{tabular}{cccc}
		\hhline{~~--}
		 & \multicolumn{1}{c|}{} & \multicolumn{1}{c}{\cellcolor[gray]{0.9}$x_1$} & \multicolumn{1}{c}{$x_2$} \\ \hhline{~---}
		\multicolumn{1}{c|}{} & \multicolumn{1}{c|}{0} & \multicolumn{1}{l}{\cellcolor[gray]{0.9}-2000} & \multicolumn{1}{l}{-3000} \\ \hhline{----}
		\multicolumn{1}{|c|}{$u_1$} & \multicolumn{1}{c|}{12} & \cellcolor[gray]{0.9}2 & 2 \\ 
		\multicolumn{1}{|c|}{\cellcolor[gray]{0.9}$u_2$} & \multicolumn{1}{c|}{\cellcolor[gray]{0.9}16} & 4 & \cellcolor[gray]{0.9}0\\ 
		\multicolumn{1}{|c|}{$u_3$} & \multicolumn{1}{c|}{20} & \cellcolor[gray]{0.9}2 & 4\\
		\multicolumn{1}{|c|}{$u_4$} & \multicolumn{1}{c|}{20} & \cellcolor[gray]{0.9}2 & 0\\
		\multicolumn{4}{c}{$B^+ = \{u_1, u_2, u_3, u_4\}$} 
	\end{tabular}
	\setlength{\extrarowheight}{0pt}
	\quad
	x^* = 
	\begin{pmatrix}
	x_1\\
	x_2
	\end{pmatrix}
	=
	\begin{pmatrix}
	0\\
	0
	\end{pmatrix}
	\quad
	\quad
	\setlength{\extrarowheight}{2pt}
	\begin{tabular}{ccrc}
		\hhline{~~--}
		& \multicolumn{1}{c|}{} & \multicolumn{1}{c}{$u_2$} & \multicolumn{1}{c}{$\cellcolor[gray]{0.9}x_2$} \\ \hhline{~---}
		\multicolumn{1}{c|}{} & \multicolumn{1}{c|}{8000} & \multicolumn{1}{l}{500} & \multicolumn{1}{l}{\cellcolor[gray]{0.9}-3000} \\ \hline
		\multicolumn{1}{|c|}{\cellcolor[gray]{0.9}$u_1$} & \multicolumn{1}{c|}{\cellcolor[gray]{0.9}4} & \cellcolor[gray]{0.9}$-\frac{1}{2}$ & 2 \\ 
		\multicolumn{1}{|c|}{$x_1$} & \multicolumn{1}{c|}{4} & $\frac{1}{4}$ & \cellcolor[gray]{0.9}0\\ 
		\multicolumn{1}{|c|}{$u_3$} & \multicolumn{1}{c|}{12} & $-\frac{1}{2}$ & \cellcolor[gray]{0.9}4\\
		\multicolumn{1}{|c|}{$u_4$} & \multicolumn{1}{c|}{12} & $-\frac{1}{2}$ & \cellcolor[gray]{0.9}0\\
		\multicolumn{4}{c}{$B^+ = \{u_1, u_3\}$} 
	\end{tabular}
	\setlength{\extrarowheight}{0pt}
	\quad
	x^* = 
	\begin{pmatrix}
	x_1\\
	x_2
	\end{pmatrix}
	=
	\begin{pmatrix}
	4\\
	0
	\end{pmatrix}	
	\]
	\ \\
	\[
	\setlength{\extrarowheight}{2pt}
	\begin{tabular}{ccrc}
	\hhline{~~--}
	& \multicolumn{1}{c|}{} & \multicolumn{1}{c}{\cellcolor[gray]{0.9}$u_2$} & \multicolumn{1}{c}{$u_1$} \\ \hhline{~---}
	\multicolumn{1}{c|}{} & \multicolumn{1}{c|}{14000} & \multicolumn{1}{l}{\cellcolor[gray]{0.9}-250} & \multicolumn{1}{l}{1500} \\ \hline
	\multicolumn{1}{|c|}{$x_2$} & \multicolumn{1}{c|}{2} & \cellcolor[gray]{0.9}$-\frac{1}{4}$ & $\frac{1}{2}$ \\ 
	\multicolumn{1}{|c|}{$x_1$} & \multicolumn{1}{c|}{4} & \cellcolor[gray]{0.9}$\frac{1}{4}$ & 0\\ 
	\multicolumn{1}{|c|}{\cellcolor[gray]{0.9}$u_3$} & \multicolumn{1}{c|}{\cellcolor[gray]{0.9}4} & $\frac{1}{2}$ & \cellcolor[gray]{0.9}-2\\
	\multicolumn{1}{|c|}{$u_4$} & \multicolumn{1}{c|}{12} & \cellcolor[gray]{0.9}$-\frac{1}{2}$ & 0\\
	\multicolumn{4}{c}{$B^+ = \{x_1, u_3\}$} 
	\end{tabular}
	\setlength{\extrarowheight}{0pt}
	\quad
	x^* = 
	\begin{pmatrix}
	x_1\\
	x_2
	\end{pmatrix}
	=
	\begin{pmatrix}
	4\\
	2
	\end{pmatrix}
	\quad
	\quad
	\setlength{\extrarowheight}{2pt}
	\begin{tabular}{ccrr}
	\cline{3-4}
	& \multicolumn{1}{c|}{} & \multicolumn{1}{c}{$u_3$} & \multicolumn{1}{c}{$u_1$} \\ \cline{2-4}
	\multicolumn{1}{c|}{} & \multicolumn{1}{c|}{16000} & \multicolumn{1}{l}{500} & \multicolumn{1}{l}{14000} \\ \hline
	\multicolumn{1}{|c|}{$x_2$} & \multicolumn{1}{c|}{4} & $\frac{1}{2}$ & $-\frac{1}{2}$ \\ 
	\multicolumn{1}{|c|}{$x_1$} & \multicolumn{1}{c|}{2} & $-\frac{1}{2}$ & 1\\ 
	\multicolumn{1}{|c|}{$u_2$} & \multicolumn{1}{c|}{8} & 2 & -4\\
	\multicolumn{1}{|c|}{$u_4$} & \multicolumn{1}{c|}{16} & 1 & 2
	\end{tabular}
	\setlength{\extrarowheight}{0pt}
	\quad
	x^* = 
	\begin{pmatrix}
	x_1\\
	x_2
	\end{pmatrix}
	=
	\begin{pmatrix}
	2\\
	4
	\end{pmatrix}	
	\]
	\ \\
	Daraus folgt, dass Produkt $x_1$ 2 mal und Produkt $x_2$ 4 mal hergestellt werden m"ussen um einen maximalen Gewinn von 16000 Euro zu erzielen.
	\newpage
	\textbf{Aufgabe 04}
	\begin{compactenum} [(a)]
		\item 
		Nach Hinzuf"ugen von Schlupfvariablen erhalten wir die folgende Matrix:\\
		\[
		A:=
		\begin{pmatrix}
		1 & 1 & 1 & 0 & 0\\
		2 & 1 & 0 & 1 & 0\\
		2 & 5 & 0 & 0 & 1
		\end{pmatrix} \quad b:= 
		\begin{pmatrix}
		10\\
		18\\
		35
		\end{pmatrix}
		\]
		\ \\
		\[
		\setlength{\extrarowheight}{2pt}
		\begin{tabular}{cccc}
			\hhline{~~--}
			& \multicolumn{1}{c|}{}    & \multicolumn{1}{c}{\cellcolor[gray]{0.9}$x_1$} & \multicolumn{1}{c}{$x_2$} \\ \hhline{~---}
			\multicolumn{1}{c|}{}    & \multicolumn{1}{c|}{-3} & \multicolumn{1}{c}{\cellcolor[gray]{0.9}-7}   & \multicolumn{1}{c}{-7} \\ \hline
			\multicolumn{1}{|c|}{$u_1$} & \multicolumn{1}{c|}{10} & \cellcolor[gray]{0.9}1 & 1\\
			\multicolumn{1}{|c|}{\cellcolor[gray]{0.9}$u_2$} & \multicolumn{1}{c|}{\cellcolor[gray]{0.9}18} & 2 & \cellcolor[gray]{0.9}1\\
			\multicolumn{1}{|c|}{$u_3$} & \multicolumn{1}{c|}{35} & \cellcolor[gray]{0.9}2 & 5\\
			\multicolumn{4}{c}{$B^+ = \{u_1, u_2, u_3\}$} 
		\end{tabular}
		\setlength{\extrarowheight}{0pt}
		\quad
		x^* = 
		\begin{pmatrix}
		x_1\\
		x_2
		\end{pmatrix}
		=
		\begin{pmatrix}
		0\\
		0
		\end{pmatrix}
		\quad
		\quad
		\setlength{\extrarowheight}{2pt}
		\begin{tabular}{ccrr}
			\hhline{~~--}
			& \multicolumn{1}{c|}{}    & \multicolumn{1}{r}{$u_2$} & \multicolumn{1}{r}{\cellcolor[gray]{0.9}$x_2$} \\ \hhline{~---} 
			\multicolumn{1}{c|}{}    & \multicolumn{1}{c|}{60} & \multicolumn{1}{r}{$\frac{7}{2}$} & \multicolumn{1}{r}{\cellcolor[gray]{0.9}$-\frac{7}{2}$} \\ \hline
			\multicolumn{1}{|c|}{\cellcolor[gray]{0.9}$u_1$} & \multicolumn{1}{c|}{\cellcolor[gray]{0.9}1} & \cellcolor[gray]{0.9}$-\frac{1}{2}$ & $\frac{1}{2}$ \\
			\multicolumn{1}{|c|}{$x_1$} & \multicolumn{1}{c|}{9} & $\frac{1}{2}$ & \cellcolor[gray]{0.9}$\frac{1}{2}$ \\
			\multicolumn{1}{|c|}{$u_3$} & \multicolumn{1}{c|}{17} & -1 & \cellcolor[gray]{0.9}4\\
			\multicolumn{4}{c}{$B^+ = \{u_1, x_1, u_3\}$}
		\end{tabular}
		\setlength{\extrarowheight}{0pt}
		\quad
		x^* = 
		\begin{pmatrix}
		x_1\\
		x_2
		\end{pmatrix}
		=
		\begin{pmatrix}
		9\\
		0
		\end{pmatrix}
		\]
		\\\
		\[
		\setlength{\extrarowheight}{2pt}
		\begin{tabular}{cccc}
		\cline{3-4}
		& \multicolumn{1}{c|}{}    & \multicolumn{1}{c}{$u_2$} & \multicolumn{1}{c}{$u_1$} \\ \cline{2-4}
		\multicolumn{1}{c|}{}    & \multicolumn{1}{c|}{67} & \multicolumn{1}{c}{0}   & \multicolumn{1}{c}{7} \\ \hline
		\multicolumn{1}{|c|}{$x_2$} & \multicolumn{1}{c|}{2} & -1 & 2\\
		\multicolumn{1}{|c|}{$x_1$} & \multicolumn{1}{c|}{8} & 1 & -1\\
		\multicolumn{1}{|c|}{$u_3$} & \multicolumn{1}{c|}{9} & 3 & -8
		\end{tabular}
		\setlength{\extrarowheight}{0pt}		
		\quad
		x^* = 
		\begin{pmatrix}
		x_1\\
		x_2
		\end{pmatrix}
		=
		\begin{pmatrix}
		8\\
		2
		\end{pmatrix}
		\quad
		\text{Damit ist die Tabelle optimal, der Wert der Zielfunktion ist 67.}
		\]
		\\\\\ 
		\item
		Nach Hinzuf"ugen von Schlupfvariablen erhalten wir die folgende Matrix:\\
		\[
		A:=
		\begin{pmatrix}
		-4 & 1 & 1 & 0 & 0\\
		1 & -1 & 0 & 1 & 0\\
		2 & -1 & 0 & 0 & 1
		\end{pmatrix} \quad b:= 
		\begin{pmatrix}
		4\\
		4\\
		10
		\end{pmatrix}
		\]
		\ \\
		\[
		\setlength{\extrarowheight}{2pt}
		\begin{tabular}{cccc}
		\hhline{~~--}
		& \multicolumn{1}{c|}{}    & \multicolumn{1}{c}{$x_1$} & \multicolumn{1}{c}{\cellcolor[gray]{0.9}$x_2$} \\ \hhline{~---}
		\multicolumn{1}{c|}{}    & \multicolumn{1}{c|}{0} & \multicolumn{1}{c}{-5}   & \multicolumn{1}{c}{\cellcolor[gray]{0.9}-4} \\ \hline
		\multicolumn{1}{|c|}{\cellcolor[gray]{0.9}$u_1$} & \multicolumn{1}{c|}{\cellcolor[gray]{0.9}4} & \cellcolor[gray]{0.9}-4 & 1\\
		\multicolumn{1}{|c|}{$u_2$} & \multicolumn{1}{c|}{4} & 1 & \cellcolor[gray]{0.9}-1\\
		\multicolumn{1}{|c|}{$u_3$} & \multicolumn{1}{c|}{10} & 2 & \cellcolor[gray]{0.9}-1\\
		\multicolumn{4}{c}{$B^+ = \{u_1\}$}
		\end{tabular}
		\setlength{\extrarowheight}{0pt}
		\quad
		x^* = 
		\begin{pmatrix}
		x_1\\
		x_2
		\end{pmatrix}
		=
		\begin{pmatrix}
		0\\
		0
		\end{pmatrix}
		\quad
		\quad
		\setlength{\extrarowheight}{2pt}
		\begin{tabular}{cccc}
		\hhline{~~--}
		& \multicolumn{1}{c|}{}    & \multicolumn{1}{l}{\cellcolor[gray]{0.9}$x_1$} & \multicolumn{1}{c}{$u_1$} \\ \hhline{~---} 
		\multicolumn{1}{c|}{}    & \multicolumn{1}{c|}{16} & \multicolumn{1}{c}{\cellcolor[gray]{0.9}-21} & \multicolumn{1}{c}{4} \\ \hline 
		\multicolumn{1}{|c|}{$x_2$} & \multicolumn{1}{c|}{4} & \cellcolor[gray]{0.9}-4 & 1 \\
		\multicolumn{1}{|c|}{$u_2$} & \multicolumn{1}{c|}{8} & \cellcolor[gray]{0.9}-3 & 1 \\
		\multicolumn{1}{|c|}{$u_3$} & \multicolumn{1}{c|}{14} & \cellcolor[gray]{0.9}-2 & 1\\
		\multicolumn{4}{c}{$B^+ = \{\varnothing\}$}
		\end{tabular}
		\setlength{\extrarowheight}{0pt}
		\]
		Da $B^+ = \{\varnothing\}$ ist kein Pivotelement w"ahlbar und das Problem somit nicht l"osbar.
	\end{compactenum} 
\end{document}
